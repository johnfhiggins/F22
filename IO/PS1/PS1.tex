%%%%%%%%%%%%%%%%%%%%%%%%%%%%%%%%%%%%%%%%%%%%%%%%%%%%%%%%%%%%%%%%%%%%%%%%%%%%%%%%%%%%
%Do not alter this block of commands.  If you're proficient at LaTeX, you may include additional packages, create macros, etc. immediately below this block of commands, but make sure to NOT alter the header, margin, and comment settings here. 
\documentclass[12pt]{article}
 \usepackage[margin=1in]{geometry} 
\usepackage{amsmath,amsthm,amssymb,amsfonts, enumitem, fancyhdr, color, hyperref,comment, graphicx, environ,mathtools, bbm, tikz, setspace, cleveref,listings, dcolumn}
\usepackage{array, multirow, caption, booktabs}
\usepackage{ mathrsfs }
\usetikzlibrary{matrix,positioning}
\tikzset{bullet/.style={circle,draw=black,inner sep=8pt}}
\DeclareMathOperator*{\argmax}{arg\,max}
\DeclareMathOperator*{\argmin}{arg\,min}
\DeclareMathOperator*{\Var}{\text{Var}}
\DeclareMathOperator*{\Cov}{\text{Cov}}

\DeclarePairedDelimiter\norm{\lVert}{\rVert}%
\newtheorem{theorem}{Theorem}
\newtheorem{lemma}[theorem]{Lemma}
\DeclareMathOperator{\eps}{\varepsilon}
\doublespacing
\DeclarePairedDelimiter\abs{\lvert}{\rvert}%
\pagestyle{fancy}
\setlength{\headheight}{65pt}
\newenvironment{problem}[2][Problem]{\begin{trivlist}
\item[\hskip \labelsep {\bfseries #1}\hskip \labelsep {\bfseries #2.}]}{\end{trivlist}}
\newenvironment{sol}
    {\emph{Solution:}
    }
    {
    \qed
    }


%%%%%%%%%%%%%%%%%%%%%%%%%%%%%%%%%%%%%%%%%%%%%%%%%%%%%%%%%%%%%%%%%%%%%%%%%%%%%%%%%


\usepackage{xcolor}
 
 


%%%%%%%%%%%%%%%%%%%%%%%%%%%%%%%%%%%%%%%%%%%%%

\rhead{John Higgins\\Econ 761 \\ 20 September, 2022} 

%%%%%%%%%%%%%%%%%%%%%%%%%%%%%%%%%%%%%%%%%%%%%


%%%%%%%%%%%%%%%%%%%%%%%%%%%%%%%%%%%%%%

\begin{document}

\begin{problem}{1}
\end{problem}
\begin{sol}
Consider a market in which goods are homogenous. 
\begin{enumerate}[label=\alph*)]
    \item Let $P(\cdot)$ be an inverse demand function with constant elasticity. This notably implies that, $\forall Q$,
    \[\eps(P(Q)) = -\frac{P(Q)}{P'(Q)Q} = -\sigma\]
    for some $\sigma >0$. Rearranging this expression, we obtain:
    \[P(Q) = \sigma P'(Q)Q \iff P(Q) - \sigma P'(Q)Q = 0\]
    Applying the Implicit Function Theorem to the above expression, we have that
    \[P'(Q) - \sigma(P'(Q) + P''(Q)Q ) = 0\]
    Hence,
    \[P'(Q) + P''(Q) Q = \frac{P'(Q)}{\sigma} > 0\]
    since both $P'(Q) > 0$ and $\sigma > 0$. It thus follows that, $\forall Q$, $P'(Q) + P''(Q) Q > 0$ for all inverse demand functions $P(\cdot)$ with constant elasticity. As such, assumption A1 of the Cournot model is violated.
    \item Consider the Cournot model with $N$ firms that each have identical cost functions $C(\cdot)$. Suppose that assumptions A1 and A2 hold. Namely, A1 requires that $0 \geq P''(Y)y_i + P'(Y)$ $\forall y_i < Y$. A2 requires that $ 0 \geq P'(Y) - C''(y_i)$ $\forall y_i \leq Y$. We wish to show that these two assumptions imply that the equilibrium price and firm quantities are decreasing in $N$.
    
    We begin by deriving each firm's profit function. Firm $i$'s profits when producing $y_i$ and facing aggregate output by opponents $Y_{-i} = \sum_{j \neq i} y_j$ are given by
    \[\pi_i(y_i, Y_{-i}) = P(y_i + Y_{-i}) y_i - C(y_i)\]
    The first order condition for $y_i$ is the following:
    \begin{align*}
        \frac{\partial \pi_i(y_i, Y_{-i})}{\partial y_i} = 0 &\iff P'(y_i + Y_{-i}) y_i + P(y_i + Y_{-i}) - C'(y_i) = 0\\
        &\iff y_i = \frac{C'(y_i) - P(y_i + Y_{-i})}{P'(y_i + Y_{-i})}
    \end{align*}
    Based on the above first order condition, we define $R(Y_{-i}) = \frac{C'(y_i) - P(y_i + Y_{-i})}{P'(y_i + Y_{-i})}$ as the best reply function of firm $i$ to opponent action profile $Y_{-i}$. 
    
    We must verify the optimality $R(Y_{-i})$ when facing opponent action profile $Y_{-i}$ using the second derivative test. The second derivative of the profit function is given by
    \[\frac{\partial^2 \pi_i(y_i, Y_{-i})}{\partial y_i^2} = P''(y_i + Y_{-i}) y_i + P'(y_i + Y_{-i}) + P'(y_i + Y_{-i}) - C''(y_i)\] 
    Note by A1, $P''(y_i + Y_{-i})y_i + P'(y_i + Y_{-i}) \leq 0$. Similarly, by A2, $P'(y_i + Y_{-i} - C''(y_i)) \leq 0$. Jointly, this implies that
    \[\frac{\partial^2 pi_i(y_i, Y_{-i})}{\partial y_i^2} = P''(y_i + Y_{-i}) y_i + P'(y_i + Y_{-i}) + P'(y_i + Y_{-i}) - C''(y_i) \leq 0\]
    Thus, firm $i$'s objective function is locally concave in $y_i$ when $y_i$ satisfies the first order condition, implying that the (unique) solution to the FOC yields a maximum of the profit function. This verifies the optimality (and uniqueness) of the best reply function $R_i(Y_{-i}$.

    We now must find the Nash equilibrium. Since each firm has identical cost functions, their best reply functions are identical. We therefore suppress the $i$ subscript and instead write $R_{i}(Y_{-i}) = R(Y_{-i})$ for $i=1,\ldots, N$. Given the symmetry of best response functions, we search for a symmetric equilibrium. That is, we suppose that $y_i = y_{j} = y^*$ $\forall i,j = 1,\ldots,N$. In order for $Y = (y^*, \ldots, y^*)$ to constitute a Nash equilibrium, we must have that 
    \[y^* = R((Y_{-i}^*) = R((N-1)y^*)\]
    Now that we have obtained the equation which characterizes the Nash equilibrium, we can analyze the impact of the number of firms $N$ on the equilibrium quantities $y^*$. 
    
    We note that the first order condition implies that
    \[\frac{\partial \pi_i(R(Y_{-i}), Y_{-i})}{\partial y_i} = 0\]
    Using the Implicit Function Theorem, we obtain that
    \[R'(Y_{-i}) = -\frac{\partial^2 \pi_i(R(Y_{-i}))/\partial Y_{-i} \partial y_i}{\partial^2 \pi_i(R(Y_{-i}), Y_{-i})/\partial y_i^2}\]
    Thus,
    \[R'(Y_{-i}) = -\frac{P'(y_i + Y_{-i}) + y_i P''(y_i + Y_{-i})}{y_i P''(y_i + Y_{-i}) + 2 P'(y_i + Y_{-i}) - C''(y_i)}\]
    Note that A1 directly implies the numerator is negative. Furthermore, we have previously demonstrated how A1 and A2 jointly imply that the denominator is negative. Together, it follows that $R'(Y_{-i}) \leq 0$. It now remains to show that $R'(Y_{-i}) \geq -1$. Note that
    \begin{align*}
        R'(Y_{-i}) &= -\frac{P'(y_i + Y_{-i}) + y_i P''(y_i + Y_{-i})}{y_i P''(y_i + Y_{-i}) + 2 P'(y_i + Y_{-i}) - C''(y_i)}\\
        &\geq -\frac{P'(y_i + Y_{-i}) + y_i P''(y_i + Y_{-i}) + P'(y_i + Y_{-i}) - C''(y_i)}{y_i P''(y_i + Y_{-i}) + 2 P'(y_i + Y_{-i}) - C''(y_i)}\\
        &= -1
    \end{align*} 
    where we the inequality comes from the fact that we added $P'(y_i + Y_{-i}) - C''(y_i)$ on the top, which is a negative quantity according to A2. This makes the overall expression (which is negative) increase. Finally, the top and bottom divide and we are left with 1.

    Hence, we have shown that $R'(Y_{-i}) \in (-1, 0)$. This implies that the best reply function is a contraction mapping and thus that the Nash equilibrium exists, is unique, and is globally stable! We now consider the implications of this for the equilibrium prices and quantities. The Nash equilibrium is characterized by the equation $y^* = R((N-1)y^*)$, where $y_i = y^*$ $\forall i=1,\ldots, N$ (such an solution is guaranteed to exist by the above argument). Applying the fact that $R'(Y_{-i}) < 0$, it follows that
    \[R'((N-1)y^*) \leq 0\]
    This implies that the equilibrium per-firm quantity $y^*$ is decreasing in the number of firms $N$. It finally remains to show that the equilibrium price is decreasing. Let $y^*_N$ indicate the per-firm equilibrium quantity with $N$ firms. We claim that $Ny^*_N$ is increasing in $N$.

    To do so, we define the mapping 
    \[\varphi : Y_{-i} \mapsto (R(Y_{-i}) + Y_{-i}) \left(\frac{N-1}{N}\right)\]
    It is clear that $\varphi(Y_{-i}, N)$ is strictly increasing in $N$, as $\frac{N-1}{N} = 1 - \frac{1}{N}$ is strictly increasing in $N$. It is also strictly increasing in $Y_{-i}$, as 
    \[\varphi'(Y_{-i}) = (R'(Y_{-i}) + 1)\left( \frac{N-1}{N}\right) \geq (-1 + 1) \frac{N-1}{N} = 0\]
    Combining these arguments, we note that $\varphi$ has increasing differences in $Y_{-i}$ and $N$ and is therefore supermodular. 
    
    Furthermore, we observe that $(N-1)y^*_{N}$ is a fixed point of $\varphi$. Indeed,
    \[\varphi((N-1)y^*_N)) = (y^*_N + (N-1)y^*_N)\left(\frac{N-1}{N}\right) = N y^*_N \left( \frac{N-1}{N}\right) = y^*_N (N-1)\]
    
    Now for the punchline! As $\varphi$ is supermodular in $N$ and $Y_{-i}$ and is increasing in $N$, its fixed points will also be increasing in $N$. Hence, $(N-1) y^*_N$ is increasing in $N$. It follows that total industry output is increasing in $N$, even though per-firm outputs decrease. This works because even though each firm produces less as the number of firms increases, the reduction in quantity is outweighed by the fact that an additional firm is producing. Hence, total industry output increases. 

    Since total industry output is increasing in $N$, it follows that $P(Ny^*)$ is decreasing in $N$ (as $P'(\cdot) < 0)$.
\end{enumerate}
\end{sol}
\begin{problem}{2}
\end{problem}
\begin{sol}
\begin{enumerate}[label=\alph*) ]
    \item The game has players $I = \{1, 2\}$. Players each have valuation $V > 0$, and this valuation is common knowledge. Players submit bids $b_i \in B_i = [0, -\infty)$. The joint strategy space is $B = B_1 \times B_2$. Player $i$'s utility function is a mapping $u : B \rightarrow \mathbb{R}$ given by
    \[u_i(b_i, b_{-i})= \begin{cases} V - b_i \quad&\text{if } b_i > b_{-i}\\
        (V - b_i)/2 \quad&\text{if } b_i = b_{-i}\\
        0 \quad&\text{if } b_i < b_j
    \end{cases}\]
    where $b_{-i}$ indicates the bid of the other player. I assume that if players tie, they each get a 50 percent chance of winning.

    We want to find the Nash equilibrium of this game. We begin by noting that, regardless of what the other player plays, playing $b_i > V$ is strictly dominated by playing $b_i = 0$. That is, $\forall i$, $\forall b_{-i}$, $\forall b_i > V$, $u(b_i, b_{-i}) < 0 = u(0, b_{-i})$. As such, no equilibrium can contain $b_i > V$ for either player.

    We now consider equilibria of the form $(b_i, b_{-i})$ where $b_i, b_{-i} \in (0, V)$. Suppose first that $b_i = b_{-i}$. We argue that this is not an equilibrium, since player $i$ could instead bid $b_i + \eps$ for some small $\eps > 0$ and be certain to win. Their payoff from bid $b_i$ is $(V - b_i)/2$, whereas their payoff from bid $b_i + \eps$ is $V - b_i - \eps$, which exceeds the tie payoff for sufficiently small $\eps$. This means that player $i$ could strictly improve by deviating, implying that this cannot be an equilibrium.

    As a second case, suppose without loss of generality that $0 < b_i < b_{-i} < V$. We argue that this cannot be an equilibrium since player $i$ would achieve a strictly higher payoff by bidding $b_i' = b_{-i} + \eps$ for $\eps < V - b_{i}$. As such, this cannot be an equilibrium.

    Now consider equilibria of the form $(b_i, 0)$ for some player $i$. This is also not an equilibrium, since $i$ could strictly benefit by playing $b_i = \eps$ for some $\eps \in (0,b_i)$ and achieve a strictly higher payoff.

    This leaves us with only $b_i = b_{-i} = V$ remaining. This gives expected payoff of 0 to both players. To see why this is a Nash equilibrium, we note that either player cannot strictly improve by playing $b_i < V$, as this would guarantee they lose and give them 0 payoff. Furthermore, bidding over $V$ is strictly dominated. Thus, neither player has a unilateral incentive to deviate, implying that this is a Nash equilibrium.
    \item We now assume the seller uses an all-pay auction. The payoffs to player 1 for the pair of bids $(b_1, b_2)$ are given by 
    \[u_1(b_1, b_2) = \begin{cases} V - b_1 &\quad\text{if } b_1 > b_2\\
        V/2 - b_1&\quad\text{if } b_1 = b_2\\
        - b_1 &\quad\text{if } b_1 < b_2
    \end{cases}\]
    As before, we assume that agents have an equal chance of winning if they submit tied bids.
    \item We argue that no pure strategy Nash equilibrium of the all pay auction exists. To do so, we proceed by considering a number of cases.
    
    As in the first price auction, playing $b_i > V$ is strictly dominated for all players and will never be played in equilibrium. 

    First consider the case where $b_i, b_j \in (0,V]$ with $b_i < b_j$. Player $i$'s payoff is $-b_i$ because they do not win the auction. They could strictly improve by playing $b_i = 0$, which gives them payoff 0.

    We now consider $b_i = b_j \in [0,V]$. Each player has expected payoff of $V/2 - b_i$. If $b_i > V/2$, $i$'s payoff will be negative, implying that $b_i > V/2$ is strictly dominated by playing $b_i' = 0$ (which gives payoff zero). Thus, an equilibrium could only consist of $b_i = b_j \leq V/2$. We argue that this is not an equilibrium, since $b_i$ could strictly benefit by playing $b_i' = b_j + \eps$ for $\eps \leq \min\{V/2, V-b_j\}$, which allows them to win the auction and have strictly higher payoff than $V/2 - b_i$. In summary, we cannot have $b_i = b_j \in [0,V]$.

    Now consider equilibria of the form $(b_i, 0)$, where $b_i \in (0, V]$. Player $i$ will win the auction and get payoff $V - b_i$. However, they could strictly improve by bidding $b_i' = \eps > 0$, which allows them to still win while paying less (so long as $\eps < b_i$). This implies that $(b_i, 0)$ with $b_i > 0$ cannot be an equilibrium.
    
    The above cases exhaust the possible pure strategy equilibrium candidates, and thus we conclude that there is no pure strategy Nash equilibrium.

    \item We wish to find the mixed strategy Nash equilibrium. To do so, we assume that each player plays with a bidding strategy $G_i(b)$, where $G_i(\cdot) = P(b_i \leq b)$ is the CDF corresponding to the distribution of player $i$'s bids. Since each player has the same valuation $V$ and this valuation is common knowledge, we impose the restriction that $G_i(b) = G_j(b) = G(b)$. That is, players play with a common bid function. 
    
    Furthermore, given the linear form of the payoff function, we conjecture that $G$ has the form $G(b) = \alpha b + \beta$ for some constants $\alpha, \beta$ to be determined. It follows that $g(b) = \alpha$ is the pdf of a player's bids. Note that this implies that $b_i = b_j$ with probability zero and we can thus neglect this possibility in players' expected utility. 
    
    Player $1$'s expected payoff from bidding strategy $G(\cdot)$ is given by
    \begin{align*}E[u(b_1, b_{j})] &= \int_{0}^{V} \int_{b_1}^V (V-b_1)\alpha \, d b_2  \alpha \, d b_1 + \int_{0}^{V} \int_{0}^{b_1} (-b_1)\alpha \, d b_2  \alpha \, d b_1 \\
        &= \alpha^2 \int_{0}^{V} (V - b_1)^2 \, d b_1 - \alpha^2 \int_{0}^V b_1^2 \, d b_1\\
        &= \alpha^2 \left[ V^3 - V^2 + \frac{V^3}{3}\right] - \alpha^2 \frac{V^3}{3}\\
        &= \frac{\alpha V^3}{3} - \alpha^2 \frac{V^3}{3}\\
        &= 0
    \end{align*}
    (player 2's expected payoff is of course identical). It makes sense that the expected value is zero: by the Revenue Equivalence Theorem, we know that any bidder's expected value will be the same regardless of auction type (as long as they are normal auctions). The first price auction considered in part a) has the same expected payoff (0), so it is encouraging that these align.

    It now remains to solve for $\alpha$ and $\beta$. Since playing $b_i > V$ is a dominated strategy, we must have that $G(V) = 1$. Hence,
    \[\alpha V + \beta = 1 \iff \beta = 1 - \alpha V\]
    We now must find $\alpha$. Given that no player can submit a negative bid, $G$ must also obey the condition that $G(0) = 0$. As such,
    \[G(0) = 0 \iff \alpha (0) + \beta = 0 \iff 1 - \alpha V = 0 \iff \alpha = \frac{1}{V}\]
    Consequently, $\beta = 1 - \alpha V = 1- \alpha \frac{1}{\alpha} = 0$. It follows that $G(b) = \frac{b}{V}$ is the equilibrium bidding strategy.
    \item We will directly compute the expected revenue to the seller (although we already know by the Revenue Equivalence Theorem that it will be equal to the expected revenue of the auction in part a), which is $V$). To do so, we simply take the expectation of the bids of each player given that they are playing with the bid function $G(\cdot)$ found in the previous section.
    
    As $G(b) = \frac{b}{V}$ is the cdf of a uniform random variable on the interval $[0,V]$, the expectation of each player's bid is $\frac{V}{2}$. Adding the expected revenue from each player, we see that the expected revenue to the seller is $\frac{V}{2} + \frac{V}{2} = V$. This aligns with the auction in part a, which has the same expected revenue (due to the RET).
\end{enumerate}
\end{sol}
\begin{problem}{3}
\end{problem}
\begin{sol}
\begin{enumerate}[label=\alph*) ]
    \item We first specify the firm-specific demand function. The demand for firm $i$ given prices $p_i, p_j$ and capacities $k_i, k_j$ is given by
    \begin{align*}
        D_i(p_i, p_j) = \begin{cases} \min\{k_i, 1\} \quad&\text{if } p_i < p_j\\
            \min\{k_i/(k_i + k_j), k_i\} \quad&\text{if } p_i = p_j\\
            \min\{k_i, 1- k_j\} \quad&\text{if } p_i > p_j
        \end{cases}
    \end{align*}
    
    We proceed by considering the following cases:
    \begin{itemize}
        \item Region I: $k_1, k_2 \geq 1$
        \item Region II: $k_1, k_2 \in [0,1]$, $k_1 + k_2 \leq 1$
        \item Region III: $k_1, k_2 \in [0,1]$, $k_1 + k_2 > 1$
        \item Region IV: $k_1 \in [0,1)$, $k_2 \geq 1$ (given symmetry of payoff structures, the argument is the same if the indices are flipped)
    \end{itemize}
    Region I: We start with $k_1, k_2 > 1$. In this case, either firm can supply the entire market. I argue that $p_1 = p_2 = 0$ is the unique equilibrium given these capacities. Suppose to the contrary that one firm sets a positive price. The other firm could simply undercut by $\eps$ and capture the entire market. Consequently, only $p_1 = p_2 = 0$ is a Nash equilibrium.

    Region II: Now suppose $k_1, k_2 \in [0,1]$ and $k_1 + k_2 \leq 1$. In this case, firms are essentially not competing with each other over any consumers. Together, they can cover the entire market demand, yet neither firm can supply more than the entire market. I argue that $p_i = p_j = 1$ is the unique equilibrium. Suppose not. Specifically, suppose $p_i < 1$ and $p_j \in [0,1]$. Firm $i$ gets demand $k_i$, giving profit $p_i k_i$. If they increase $p_i$ to $p_i' = p_j$, they would get demand of $\min\{k_i /(k_i + k_j), k_i\} = k_i$ (since $k_i + k_j \leq 1$, implying $\frac{k_i}{k_i + k_j} \geq k_i$). Hence, their profit is $p_j k_i > p_i k_i$, which is strictly higher. Thus, they would benefit from increasing $p_i$ and $p_i < 1$ cannot be an equilibrium. We are only left with $p_i = p_j = 1$. No firm would seek to deviate by lowering prices, as they get the same demand either way and would only reduce their revenue by cutting prices. Hence, in this scenario, $p_i = p_j = 1$ is the unique Nash equilibrium.

    Region III: Consider now the case where $k_1, k_2 \in [0,1]$ and $k_1 + k_2 > 1$. This implies that neither firm can completely serve the market, but together they have capacity which exceeds the market. I argue that there is no Nash equilibrium in pure strategies in this region. Suppose to the contrary that $p_i, p_j \in [0,1]$ constitutes a Nash equilibrium. In particular, first suppose $p_i = p_j$. We have that $k_i + k_j \geq 1$, implying that $\min\{k_i/(k_i + k_j), k_i\} = \frac{k_i}{k_i + k_j}$. Thus, firm $i$ has incentive to undercut $j$ by setting $p_i = p_j - \eps$ and capturing a larger share than when $p_i = p_j$. Hence, $p_i = p_j$ cannot be an equilibrium. Suppose now that $p_i < p_j$ (without loss). Firm $i$ consequently gets demand $\min\{1, k_i\}$ at price $p_i$. However, they could get the same demand if they were to set a price $p_i = p_j - \eps$ for $\eps > 0$ sufficiently small, yet the higher price would give them strictly higher revenue. Thus, $p_i < p_j$ cannot be an equilibrium either. $p_i = p_j$ and $p_i < p_j$ exhaust the possible equilibrium candidates (modulo permutation of firm index), implying that there is no equilibrium in pure strategies.

    Region IV: Finally, suppose $k_1 \in [0,1)$, $k_2 \geq 1$. Here, firm 2 can supply the entire market, whereas 1 cannot. I claim that $p_1 = 0$ and $p_2 = 1$ is the unique Nash equilibrium in this region. Firm 2 has incentive to undercut any positive price set by firm 1, meaning that the only stable scenario is where $p_1 = 0$ and $p_2 = 1$.
    \item We search for the upper and lower bound of the support for the mixed strategy equilibrium played when $k_1, k_2$ are such that $k_1, k_2 \in [0,1]$ and $k_1 + k_2 > 1$. To do so, we use the fact that $\bar{p} = 1$ ensures that the firm with larger capacity will ``lose the market'' almost surely (i.e. they will get only the residual demand). Suppose without loss that $k_1 \leq k_2$ so that firm 2 has larger capacity. Firm 1's revenue is $ \bar{p} (1-k_2) = 1-k_2$. We now search for $\underline{p}$. We know that the revenue to firm $1$ must be the same at $\underline{p}$ as it was at $\bar{p}$. When firm 1 plays $\underline{p}$ and thus wins almost surely, they will get demand $k_1$. As such, their revenue is $\underline{p} k_1$. Equating this with $1-k_2$, we obtain that
     \[1 - k_2 = \underline{p} k_1 \iff \underline{p} = \frac{1-k_2}{k_1}\]
     Hence, the support of the mixed strategy equilibrium is $\left[\frac{1-k_2}{k_1}, 1\right]$, where $k_2$ is the firm with larger capacity. If firm 1 were to have larger capacity, the indices would be flipped but the result would otherwise be unchanged.
     \item We now derive profit functions for the above regions. 
     \begin{itemize}
        \item Region I: In Region I, the equilibrium in the pricing stage is to have $p_i = p_j = 0$. Thus, firm profits are given by 
        \[\Pi_i(k_i, k_j) = - c k_i\]
        This is clearly decreasing in $k_i$ and is maximized with $k_i^* = 0$, implying that $k_i \geq 1$ cannot be optimal. Hence, this cannot be a subgame perfect equilibrium.
        \item Region II: In Region II, the equilibrium in the pricing stage was $p_i = p_j = 1$. This implies that firm profits are given by
         \[\Pi_i (k_i, k_j) = k_i - c k_i\]
         This is maximized by setting $k_i = 1- k_j$ (in order to stay in Region II). The question is: can firm i profitably deviate by increasing $k_i$ so that they go into Region III? The answer is no: we know that the expected revenue in Region III for firm $i$ is $1-k_j$, which is exactly the revenue they'd get by setting $k_i = 1- k_j$! So thus any combinations of $k_i^*, k_j^* \in [0,1]$ such that $k_i^* + k_j^* = 1$ constitute subgame-perfect Nash equilibria.
         \item Region III: In Region III, players play according to the mixed strategy specified above which gives expected revenue $1 - k_j$ to firm $i$. Firm $i$'s profits are therefore
         \[\Pi_i(k_i, k_j) = 1 - k_j - c k_i\]
         This is strictly decreasing in $k_i$, so firm $i$ will optimally set $k_i = 0$. This is inconsistent with the fact that $k_i + k_j > 1$, implying that this cannot be subgame-perfect.
         \item Finally, if $k_1 \in [0,1)$ and $k_2 \geq 1$, firm 1 gets profit of 
         \[\Pi_1(k_1, k_2) = - c k_1\]
         which is decreasing in $k_1$. Thus, firm 1 optimally chooses $k_1 =0$.  
         Firm 2 gets profits of
         \[\Pi_2(k_1, k_2) = 1 - c k_2\]
         which is optimized by setting $k_2$ as low as it can go without adversely impacting revenue. If they lower it below 1, they go into Region III and get expected revenue $1-k_1$, which is less than 1. Thus, they set $k_2 = 1$. If both $k_2 = 1 and k_1 = 0$ (or vice versa), then this is subgame perfect (and technically, it is in Region II). Any other $(k_1, k_2)$ pair in Region IV is not subgame-perfect.
     \end{itemize}

     We finally show that the profit functions have the Cournot form in the subgame perfect equilibrium. In Region 2, we have that 
     \[\Pi_i(k_i, k_j) = k_i (1-c)\]
     Suppose now that firms instead engage in Cournot competition where they choose quantities to supply (subject to the constraint that $k_i + k_j \leq 1$), and that the market price is 1. It follows that their profit functions are of the form
     \[\Pi_i (k_i, k_j) = P(k_i, k_j)k_i - c k_i = k_i - c k_i\]
     This is exactly identical to firm's profit functions in the subgame perfect equilibrium!  
\end{enumerate}
\end{sol}
\end{document}
