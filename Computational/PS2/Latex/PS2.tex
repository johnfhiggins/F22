        %%%%%%%%%%%%%%%%%%%%%%%%%%%%%%%%%%%%%%%%%%%%%%%%%%%%%%%%%%%%%%%%%%%%%%%%%%%%%%%%%%%%
%Do not alter this block of commands.  If you're proficient at LaTeX, you may include additional packages, create macros, etc. immediately below this block of commands, but make sure to NOT alter the header, margin, and comment settings here. 
\documentclass[12pt]{article}
 \usepackage[margin=1in]{geometry} 
\usepackage{amsmath,amsthm,amssymb,amsfonts, enumitem, fancyhdr, color, hyperref,comment, graphicx, environ,mathtools, bbm, tikz, setspace, cleveref,listings, dcolumn}
\usepackage{array, multirow, caption, booktabs}
\usepackage{ mathrsfs }
\usetikzlibrary{matrix,positioning}
\tikzset{bullet/.style={circle,draw=black,inner sep=8pt}}
\DeclareMathOperator*{\argmax}{arg\,max}
\DeclareMathOperator*{\argmin}{arg\,min}
\DeclareMathOperator*{\Var}{\text{Var}}
\DeclareMathOperator*{\Cov}{\text{Cov}}

\DeclarePairedDelimiter\norm{\lVert}{\rVert}%
\newtheorem{theorem}{Theorem}
\newtheorem{lemma}[theorem]{Lemma}
\DeclareMathOperator{\eps}{\varepsilon}
\doublespacing
\DeclarePairedDelimiter\abs{\lvert}{\rvert}%
\pagestyle{fancy}
\setlength{\headheight}{65pt}
\newenvironment{problem}[2][Problem]{\begin{trivlist}
\item[\hskip \labelsep {\bfseries #1}\hskip \labelsep {\bfseries #2.}]}{\end{trivlist}}
\newenvironment{sol}
    {\emph{Solution:}
    }
    {
    \qed
    }


%%%%%%%%%%%%%%%%%%%%%%%%%%%%%%%%%%%%%%%%%%%%%%%%%%%%%%%%%%%%%%%%%%%%%%%%%%%%%%%%%


\usepackage{xcolor}
 
 


%%%%%%%%%%%%%%%%%%%%%%%%%%%%%%%%%%%%%%%%%%%%%

\rhead{John Higgins\\Econ 899 \\ 21 September, 2022} 

%%%%%%%%%%%%%%%%%%%%%%%%%%%%%%%%%%%%%%%%%%%%%


%%%%%%%%%%%%%%%%%%%%%%%%%%%%%%%%%%%%%%

\begin{document}

\section{I. Huggett (1993) with enforceable insurance}
As in Hugget (1993), suppose there is a continuum of agents with total mass equal to one. Each agent has no initial asset holdings. Each period, agents receive an endowment of a consumption good. The endowment $e$ takes values in $E = \{e_h, e_l\}$, with $e_h > e_l$. 

The endowment process for each agent follows a Markov process with $\pi(e' \mid e) = P(e_{t+1} = e' \mid e_{t} = e) > 0$ for each $e, e' \in E$. We assume each realization of the endowment process is iid and thus independent of all other agents' endowment realizations.

Agents have preferences given by 
\[E\left[ \sum_{t=0}^{\infty} \beta^t \frac{c^{1-\sigma}}{1-\sigma}\right] \quad \sigma > 1\]
where $E$ is the expectation operator with respect to the stochastic endowment process.

Agents are able to purchase state-contingent insurance which pays out based on their realization of their endowment in the next period. That is, agents can buy and sell Arrow securities $a_h^{t+1}$ and $a_l^{t+1}$, where $a_h^{t+1}$ pays out one unit of the consumption good in period $t+1$ when the agent receives endowment $e_h$, and $a_l^{t+1}$ pays out one unit when the agent receives $e_l$. These time $t$ contracts have prices $q_h^{t}$ and $q_l^{t}$, respectively. We assume these contracts are enforceable and thus all agents can credibly commit to fulfilling their obligations.

Agents in time $t$ with current endowment $e_h$ and current holdings $a_h^t$ and $a_l^t$ facing prices $q_h^t$ and $q_l^t$ obey the following constraint:
\[c_t + q_l^t a_l^{t+1} + q_h^t a_h^{t+1} \leq e_h + a_h^t\]
Similarly, agents with endowment $e_l$ have the following constraint:
\[c_t + q_l^t a_l^{t+1} + q_h^t a_h^{t+1} \leq e_l + a_l^t\]
Now that we have set up the problem, we wish to find the competitive equilibrium of such an economy. We start by finding the solution to the planner's problem and then decentralizing it.

Let $\Pi_h^t$ and $\Pi_l^{t}$ indicate the proportion of agents in time $t$ with $e_h$ and $e_l$, respectively. %Since there is a continuum of agents which each realize iid endowment shocks, the Law of Large Numbers dictates that the proportion of agents with endowment $e_h$ will simply be $\Pi_h^{t+1} = \pi(e_h \mid e_h) \Pi_h^{t} + \pi(e_h \mid e_l) \Pi_l^t$. Similarly, the mass of agents with endowment $e_l$ will be $

Assuming the planner is utilitarian and weighs all agents equally, the planner solves
\begin{align*}
    \max_{\{c_h^t, c_l^t\}_{t=0}^{\infty}} \sum_{t=0}^{\infty} \beta^t \left[\Pi_h^t \frac{(c_h^t)^{1-\sigma}}{1-\sigma} + \Pi_l^t \frac{(c_l^t)^{1-\sigma}}{1-\sigma} \right] \\
    \text{s.t. } c_h^t, c_h^t \geq 0, \quad \forall t\\
    \Pi_h^t c_h^t + \Pi_l^t c_h^t \leq \Pi_h^t e_h + \Pi_l^t e_l, \quad \forall t
\end{align*}
Assuming the non-negativity constraint on consumption doesn't bind, the planner's problem has the following Lagrangian:
\[\mathcal{L} = \sum_{t=0}^{\infty} \beta^t \left[\Pi_h^t \frac{(c_h^t)^{1-\sigma}}{1-\sigma} + \Pi_l^t \frac{(c_l^t)^{1-\sigma}}{1-\sigma} \right] + \lambda_t [\Pi_h^t e_h + \Pi_l^t e_l - \Pi_h^t c_h^t - \Pi_l^t c_l^t]\]
This has the following first order conditions:
\[\beta^t \Pi_h^t (c_h^t)^{-\sigma}  \leq \Pi_h^t \lambda_t \]
\[\beta^t \Pi_l^t (c_l^t)^{-\sigma} \leq \Pi_l^t \lambda_t \]
with $\lambda_t \geq 0$ $\forall t$, as well as the complementary slackness condition:
\[\lambda_t[\Pi_h^t e_h + \Pi_l^t e_l - \Pi_h^t c_h^t - \Pi_l^t c_l^t] = 0\]
If we let the inequality constraints bind, we note that we have
\[(c_h^t)^{-\sigma} =( c_l^t)^{-\sigma} = \lambda_t > 0\]
since $c_h^t, c_l^t >0$ by our initial assumption. This means that the planner will optimally set $c_h^t = c_l^t = c^t$ $\forall t$. It now remains to find $c^t$. Note that since $\lambda_t > 0$ $\forall t$, it follows that the planner's resource constraint will bind. Hence,
\[\Pi_h^t c^t + \Pi_l^t c^t = \Pi_h^t e_h + \Pi_l^t e_l \iff c^t = \frac{\Pi_h^t e_h + \Pi_l^t e_l}{\Pi_h^t + \Pi_l^t} =\Pi_h^t e_h + \Pi_l^t e_l \]
since $\Pi_h^t + \Pi_l^t = 1$.  

Furthermore, if the Markov process is assumed to have a stationary distribution, we note that $\Pi_h^t = \Pi_h$ and $\Pi_l^t = \Pi_l$ for all $t$, for some $\Pi_h$ and $\Pi_l$ such that $\Pi_h + \Pi_l = 1$ (since there is a continuum of agents). This means that the allocation is constant across time (since the aggregate endowment is constant). Therefore, we have that 
\[\forall t, c^t = \bar{c} = \Pi_h e_h + \Pi_l e_l\]

In summary, the planner sets consumption for all agents to be the same, and the allocation depends only on the aggregate endowment which is constant across time. 

For concreteness, we use the parameterization specified in part II which has $e_h = 1$, $e_l = 0.5$, $\Pi_l = 0.06$, and $\Pi_h = 0.94$. Hence,
\[\bar{c} = 0.94 \cdot 1 + 0.06 \cdot 0.5 = 0.97\]

We now decentralize this solution using the Arrow-security setup specified previously. A competitive equilibrium involves agents maximizing their utility as described above and markets clearing. We note that, in order to implement the planner's solution, it must be the case that agents sell $a_h^{t+1}$ and buy $a_l^{t+1}$ to insure themselves against adverse shocks (since $e_l < \bar{c} < e_h$).

Let $c(e_t)$ indicate the consumption of an agent with endowment $e_t$, and $a(e_t)$ indicate their insurance purchase. The asset market clearing condition is that $\Pi_h a(e_h) = -\Pi_l a(e_l)$.

The agent's problem is thus
\[\max_{c(e_t), a(e_{t+1})} E\left[ \sum_{t=0}^{\infty} \frac{c(e_t)^{1-\sigma}}{1-\sigma}\right]\]
subject to the sequential budget constraint
\[c(e_t) + q a(e_{t+1}) \leq e_t + a(e_t)\]
Their Lagrangian is the following:
\[\mathcal{L} = E\left[\sum_{t=0}^{\infty} \beta^t\frac{c(e_t)^{1-\sigma}}{1-\sigma}\right] + \lambda_t [ e_t + a(e_{t+1}) - c(e_t) - q a(e_t)]\]
which has FOCs
\[\beta^t c(e_t)^{-\sigma} =\lambda_t \quad [c(e_t)]\]
\[\lambda_t = q \lambda_{t+1} \quad [a(e_{t+1})]\]
Combining these, we obtain
\[\beta^t c(e_t)^{-\sigma} = q \beta^{t+1} c(e_{t+1})^{-\sigma}\]
Imposing the condition from the planner's allocation that $c(e_t) = c(e_{t+1}) = \bar{c}$ $\forall t$, it follows that
\[\beta^t = q \beta^{t+1} \implies q = \beta\]
Hence, the price to a claim on consumption in the next period is given by $q= \beta$. The allocation of consumption is the same as the planner's problem. It remains to find the assets purchased by each type. We note that
\[a(e_h) = \bar{c} - e_h \quad \text{and} \quad a(e_l) = \bar{c} - e_l\]
Again using the parameterization from part II, this implies that
\[a(e_h) = -0.03 \quad \text{and} \quad a(e_l) = 0.47\]
We verify that asset market clearing is satisfied:
\[\Pi_h a(e_h) =(0.94) (-0.03) = 0.0282 = (0.06)(0.47) = -\Pi_l a(e_l) \]
\section{II. Computing Huggett with incomplete markets} 
\begin{enumerate}
    \item Hello
\end{enumerate}
\end{document}
